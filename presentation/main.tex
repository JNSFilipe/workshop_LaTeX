\documentclass[aspectratio=169]{beamer}
\usepackage{template/sps_beamer}
\usepackage{subcaption}
\usepackage{booktabs}
\usepackage{cleveref}
\usepackage{graphicx}
\usepackage{minted}
\usepackage{tikz}


\title{\LaTeX \ Workshop}
% \subtitle{Subtitle}
\author{Jose N. Filipe}
\date{4 December 2024}

% Enable Numbers in floats!
\setbeamertemplate{caption}[numbered]{}

\begin{document}
% COVER
{
% Use simple cover
% \usebackgroundtemplate{\includegraphics[width=\paperwidth]{template_background/beamer_sps_cover_simple.pdf}}
% Use simple cover with IPL logo
\usebackgroundtemplate{\includegraphics[width=\paperwidth]{template/beamer_sps_cover_simple_ipl.pdf}}
% Use cover with some images
% \usebackgroundtemplate{\includegraphics[width=\paperwidth]{template_background/beamer_sps_cover.pdf}}
\begin{frame}
    \titlepage
\end{frame}
}

% TABLE OF CONTENTS
{
\usebackgroundtemplate{\includegraphics[width=\paperwidth]{template/beamer_sps_final.pdf}}
\begin{frame}
    \frametitle{Table of Contents}
    \tableofcontents
\end{frame}

% Uncomment this to get table of contents on each section

}

\AtBeginSection[]
{
  \begin{frame}
    \frametitle{Table of Contents}
    \tableofcontents[currentsection]
  \end{frame}
}

%CONTENT


\section{Introduction}

\begin{frame}{\LaTeX: How and Why?}
    \begin{itemize}
        \item \textbf{Origins:}
            \begin{itemize}
                \item Developed by \textbf{Leslie Lamport} in 1984
                \item Built on top of Donald Knuth's \textbf{TeX} typesetting system (1978)
            \end{itemize}
        \item \textbf{Purpose:}
            \begin{itemize}
                \item Simplify the creation of complex documents
                \item Provide superior control over document layout and formatting
            \end{itemize}
        \item \textbf{Why:}
            \begin{itemize}
                \item Many packages to do what you probably want to do
                \item No need to worry about formatting: someone does the template, you just write
            \end{itemize}
    \end{itemize}
\end{frame}

\begin{frame}{\LaTeX: How and Why? What about Word?}
    \begin{itemize}
        \item \textbf{Superior Typesetting Quality:}
            \begin{itemize}
                \item Professional and consistent formatting
                \item Optimal handling of mathematical equations and symbols
            \end{itemize}
        \item \textbf{Automated Document Management:}
            \begin{itemize}
                \item Automatic numbering of sections, figures, tables, and references
                \item Easy generation of tables of contents, bibliographies, and indexes
            \end{itemize}
        \item \textbf{Version Control Compatibility:}
            \begin{itemize}
                \item Plain text files integrate seamlessly with version control systems (e.g., Git)
            \end{itemize}
        \item \textbf{Stability and Consistency:}
            \begin{itemize}
                \item Consistent output across different platforms and devices
                \item Minimal formatting issues when sharing documents
            \end{itemize}
        \item \textbf{Focus on Content:}
            \begin{itemize}
                \item Encourages separation of content and styling
                \item Reduces distractions from formatting during the writing process
            \end{itemize}
    \end{itemize}
\end{frame}

\section{Hello World!}

\begin{frame}[fragile]{The Overleaf Minimum}
    % \begin{lstlisting}[language=TeX, caption=Hello World!]
    \begin{minted}{tex}
        \documentclass{article}
        \usepackage{graphicx} % Required for inserting images

        \title{Latex workshop}
        \author{José Filipe}
        \date{December 2024}

        \begin{document}

        \maketitle

        \section{Introduction}

        \end{document}
    \end{minted}
\end{frame}

\begin{frame}[fragile]{The Bare Bones}
    % \begin{lstlisting}[language=TeX, caption=Hello World!]
    \begin{minted}{tex}
        \documentclass[12pt]{article}

        \begin{document}
            Hello world!
        \end{document}
    \end{minted}
\end{frame}

\section{The Basics}

\begin{frame}[fragile]{Paragraphs and linebreaks}
    \begin{itemize}
        \item One ``enter'' does nothing
        \item Blank line to create new paragraph
        \item \mintinline{latex}{\\} or \mintinline{latex}{\linebreak} for break a line
    \end{itemize}
\end{frame}

\begin{frame}[fragile]{Paragraphs and linebreaks}
    \begin{minted}{tex}
        \documentclass[12pt]{article}

        \begin{document}
            Lorem ipsum odor amet, consectetuer adipiscing elit.
            Donec proin hac nostra suspendisse nunc facilisis quisque.

            Faucibus metus justo varius \linebreak pretium
            erat viverra auctor.
            Habitasse quis purus iaculis \\ condimentum at
            inceptos magnis.
        \end{document}
    \end{minted}
\end{frame}


\begin{frame}[fragile]{Bold, italics, and underlines}
    \begin{itemize}
        \item \mintinline{latex}{\textbf{...}} bold text
        \item \mintinline{latex}{\textit{...}} italic text
        \item \mintinline{latex}{\underline{...}} underline text
    \end{itemize}
\end{frame}

\begin{frame}[fragile]{Bold, italics, and underlines}
    \begin{minted}{tex}
        \documentclass[12pt]{article}

        \begin{document}
            Lorem \textbf{ipsum odor} amet,
            consectetuer \textit{adipiscing} elit.

            Donec proin hac \underline{nostra} suspendisse
            nunc \textbf{\textit{\underline{facilisis quisque}}}.
        \end{document}
    \end{minted}
\end{frame}

\begin{frame}[fragile]{Generalising Commands}
    \begin{itemize}
        \item As you probably noticed by now...
        \begin{itemize}
            \item There is an escape character: \mintinline{tex}{\}
            \item Commands start with \mintinline{tex}{\}
            \begin{itemize}
                \item \textbf{\underline{Almost every direct interface with \LaTeX \ starts with the }}\mintinline{tex}{\}
                \item To write the escape character: \mintinline{tex}{\textbackslash}
            \end{itemize}
            \item Followed by the command name
            \item Optional arguments are enclosed in square brackets \mintinline{tex}{[...]}
            \item Arguments are enclosed in curly braces \mintinline{tex}{{...}}
            \item Multiple arguments are provided sequentially
        \end{itemize}
    \end{itemize}
\end{frame}

\begin{frame}[fragile]{Generalising Commands}
    \begin{centering}
        \begin{minted}{tex}
            \comandName[opt. 1, opt. 2, ...]{arg. 1}{arg. 2}...
        \end{minted}
    \end{centering}
\end{frame}

\begin{frame}[fragile]{Another \LaTeX \ interface: Environments}
    \begin{itemize}
        \item Define blocks of content with specific formatting or behaviour
        \item Start with \mintinline{latex}{\begin{environment}} and end with \mintinline{latex}{\end{environment}}
        \item The environment name must match in both \mintinline{latex}{\begin{...}} and \mintinline{latex}{\end{...}}
        \item Some environments accept optional arguments in square brackets \texttt{[...]}
        \item Common environments: \mintinline{latex}{itemize}, \mintinline{latex}{enumerate}, \mintinline{latex}{figure}, \mintinline{latex}{table}, \mintinline{latex}{equation}, etc.
    \end{itemize}
\end{frame}

\begin{frame}[fragile]{Lists}
    \begin{minted}{tex}
        \documentclass[12pt]{article}

        \begin{document}

            \begin{itemize}
                \item Lorem ipsum odor amet
                \item Justo sed duis purus
                \begin{itemize}
                    \item Consectetuer adipiscing elit
                    \item Donec proin hac nostra suspendisse
                \end{itemize}
                \item Nunc facilisis quisque
            \end{itemize}
        \end{document}
    \end{minted}
\end{frame}

\begin{frame}[fragile]{Lists}
    \begin{itemize}
        \item Lorem ipsum odor amet
        \item Justo sed duis purus
        \begin{itemize}
            \item Consectetuer adipiscing elit
            \item Donec proin hac nostra suspendisse
        \end{itemize}
        \item Nunc facilisis quisque
    \end{itemize}
\end{frame}

\begin{frame}[fragile]{Numbered Lists}
    \begin{minted}{tex}
        \documentclass[12pt]{article}

        \begin{document}

            \begin{enumerate}
                \item Lorem ipsum odor amet
                \item Justo sed duis purus
                \begin{enumerate}
                    \item Consectetuer adipiscing elit
                    \item Donec proin hac nostra suspendisse
                \end{enumerate}
                \item Nunc facilisis quisque
            \end{enumerate}
        \end{document}
    \end{minted}
\end{frame}

\begin{frame}[fragile]{Numbered Lists}
    \begin{enumerate}
        \item Lorem ipsum odor amet
        \item Justo sed duis purus
        \begin{enumerate}
            \item Consectetuer adipiscing elit
            \item Donec proin hac nostra suspendisse
        \end{enumerate}
        \item Nunc facilisis quisque
    \end{enumerate}
\end{frame}

\section{Figures, References, and Packages}

\begin{frame}[fragile]{Figures}
    \begin{minted}{tex}

        \documentclass[12pt]{article}

        \begin{document}
            \begin{figure}
                \centering
                \includegraphics[width=0.5\columnwidth]{Figures/sps_logo}
                \caption{Logo IEEE SPS.}
            \end{figure}
        \end{document}
    \end{minted}
\end{frame}

\begin{frame}[fragile]{Figures}
    \begin{itemize}
        \item Use a \mintinline{tex}{figure} environment
        \item \mintinline{tex}{\centering}:
            \begin{itemize}
                \item Centers the content within the \mintinline{tex}{figure} environment
                \item Ensures the image is horizontally aligned in the middle of the page or column
            \end{itemize}

        \item \mintinline{tex}{\includegraphics}:
            \begin{itemize}
                \item Inserts an image into the document
                \item Commonly used options:
                    \begin{itemize}
                        \item \texttt{width}: Specifies the width of the image (e.g., \mintinline{latex}{width=0.5\columnwidth})
                        \item \texttt{height}: Specifies the height of the image
                        \item \texttt{scale}: Scales the image by a factor
                \end{itemize}
            \item Takes path to image as argument (inside \mintinline{latex}{{...}})
            \end{itemize}

        \item \mintinline{latex}{\caption}:
            \begin{itemize}
                \item Adds a caption below the image
            \end{itemize}
    \end{itemize}
\end{frame}

\begin{frame}{Figures}
    \begin{figure}
        \centering
        \includegraphics[width=0.3\columnwidth]{Figures/sps_logo}
        \caption{Logo IEEE SPS.}
    \end{figure}
\end{frame}

\begin{frame}[fragile]{Figure Placement}
    \begin{itemize}
        \item LaTeX handles figure placement automatically for optimal layout
        \item Common placement specifiers:
            \begin{itemize}
                \item \mintinline{latex}{h} – here
                \item \mintinline{latex}{t} – top
                \item \mintinline{latex}{b} – bottom
                \item \mintinline{latex}{!} – override internal parameters
            \end{itemize}
        \item Messing with Figure placement can disrupt the document flow!
        \item Letting \LaTeX \ decide ensures the professional appearance
        \item Just accept that \LaTeX \ knows better, this is not MS Word and that is a bonus!
    \end{itemize}
\end{frame}

\begin{frame}[fragile]{References}
    \begin{minted}{tex}
        \documentclass[12pt]{article}
        \usepackage{graphicx}

        \begin{document}
            \begin{figure}
                \centering
                \includegraphics[width=0.5\columnwidth]{Figures/sps_logo}
                \caption{Logo IEEE SPS.}
                \label{fig:sps}
            \end{figure}

            Figure~\ref{fig:sps} is the SPS logo!
        \end{document}
    \end{minted}
\end{frame}

\begin{frame}[fragile]{References}
    \begin{itemize}
        \item \mintinline{tex}{\label{...}} assigns a label to an element (e.g., figures, tables, sections)
        \begin{itemize}
            \item We can call a figure by a name
            \item No matter the order, it will always be the right figure!
            \item It is a breeze to create lists of figure
            \item Labels for figure usually start with \mintinline{tex}{fig:}
        \end{itemize}
        \item \mintinline{tex}{\ref{...}} references the labelled element
        \item Using \mintinline{tex}{~} ensures that "Figure" and its number stay on the same line
    \end{itemize}
\end{frame}

\begin{frame}[fragile]{References}
    \begin{figure}
        \centering
        \includegraphics[width=0.3\columnwidth]{Figures/sps_logo}
        \caption{Logo IEEE SPS.}
        \label{fig:sps}
    \end{figure}

    Figure~\ref{fig:sps} is the SPS logo!
\end{frame}

\begin{frame}[fragile]{Packages and Clever References}
    \begin{minted}{tex}
        \documentclass[12pt]{article}
        \usepackage{graphicx}
        \usepackage{cleveref}

        \begin{document}
            \begin{figure}
                \centering
                \includegraphics[width=0.5\columnwidth]{Figures/sps_logo}
                \caption{Logo IEEE SPS.}
                \label{fig:sps}
            \end{figure}

            \cref{fig:sps} is the SPS logo!
        \end{document}
    \end{minted}
\end{frame}

\begin{frame}[fragile]{Packages and Clever References}
    \begin{figure}
        \centering
        \includegraphics[width=0.3\columnwidth]{Figures/sps_logo}
        \caption{It is yet another IEEE SPS logo.}
        \label{fig:sps2}
    \end{figure}

    Both Figure~\ref{fig:sps} and \cref{fig:sps2} are the SPS logo!
\end{frame}

\begin{frame}[fragile]{Packages and Clever References}
    \begin{itemize}
        \item \mintinline{tex}{\usepackage{..}} loads extra functionalities, packages made by the communities (basically, libraries)
        \item We are importing the \mintinline{tex}{cleveref} package
        \begin{itemize}
            \item \mintinline{tex}{\cref{...}} automatically includes the type of the referenced object (e.g., "Figure", "Table") without manual text
            \item Using \mintinline{tex}{\cref{fig:sps}} produces "Figure 1" automatically, maintaining proper spacing and formatting
        \end{itemize}
    \end{itemize}
\end{frame}

\section{Equations}

% New Topic: The Equation Environment
\begin{frame}[fragile]{The Equation Environment}
    \begin{itemize}
        \item Use the \mintinline{latex}{equation} environment to display numbered equations.
        \item Automatically centres the equation and assigns a number for referencing.
        \item Syntax:
            \begin{minted}{latex}
                \begin{equation}
                    % Your equation here
                \end{equation}
            \end{minted}
    \end{itemize}
\end{frame}

\begin{frame}[fragile]{The Equation Environment}
    \begin{minted}{tex}
        \documentclass[12pt]{article}

        \begin{document}
            Here is a simple equation:

            \begin{equation}
                E = mc^2
            \end{equation}
        \end{document}
    \end{minted}
\end{frame}

% New Topic: Mathematical Notation Basics
\begin{frame}[fragile]{Mathematical Notation Basics}
    \begin{itemize}
        \item Greek Letters: \mintinline{latex}{\alpha} ($\alpha$), \mintinline{latex}{\beta} ($\beta$),\mintinline{latex}{\delta} ($\delta$), \mintinline{latex}{\Delta} ($\Delta$), etc.
        \item Operations: \mintinline{latex}{+} ($+$), \mintinline{latex}{-} ($-$), \mintinline{latex}{\times} ($\times$), \mintinline{latex}{\cdot} ($\cdot$)
        \item Fractions: \mintinline{latex}{\frac{a}{b}} \ \ \ $\left ( \frac{a}{b} \right )$
        \item Subscript: \mintinline{latex}{a_{i}} \ \ \ $\left ( a_i \right )$
        \item Superscript: \mintinline{latex}{x^{2}} \ \ \ $\left ( x^2 \right )$
        \item Summations: \mintinline{latex}{\sum_{i=1}^{n} x_i} \ \ \  $\left ( \sum_{i=1}^{n} x_i \right )$
        \item Integrals: \mintinline{latex}{\int_{a}^{b} f(x) \, dx} \ \ \ $\left ( \int_{a}^{b} f(x) \, dx \right )$
    \end{itemize}
\end{frame}

\begin{frame}[fragile]{Mathematical Notation Basics -- Challenge}
    \begin{equation}
        \sigma = \sqrt{\frac{\sum_{i=1}^{N} (x_i - \mu)^2}{N}}
    \end{equation}
\end{frame}

\begin{frame}[fragile]{Referencing Equations}
    \begin{minted}{tex}
        \documentclass[12pt]{article}
        \usepackage{cleveref}

        \begin{document}
            The \cref{eq:std} shows the standard deviation:

            \begin{equation}
                \sigma = \sqrt{\frac{\sum_{i=1}^{N} (x_i - \mu)^2}{N}}
                \label{eq:std}
            \end{equation}

        \end{document}
    \end{minted}
\end{frame}

\begin{frame}[fragile]{Referencing Equations}
    The \cref{eq:std} shows the standard deviation:

    \begin{equation}
        \sigma = \sqrt{\frac{\sum_{i=1}^{N} (x_i - \mu)^2}{N}}
        \label{eq:std}
    \end{equation}
\end{frame}

\begin{frame}[fragile]{Inline Equations}
    \begin{itemize}
        \item Embed mathematical expressions within text using \mintinline{latex}{$...$}
        \item For inline equations and symbols
        \item Supports standard \LaTeX \ math syntax
    \end{itemize}
\end{frame}

\begin{frame}[fragile]{Inline Equations}
    \begin{minted}{tex}
        \documentclass[12pt]{article}

        \begin{document}
            According to Newton's second law, the force applied
            to an object is given by $F = ma$, where:
            \begin{itemize}
                \item $F$ is the force in Newtons ($N$)
                \item $m$ is the mass in kilograms ($kg$)
                \item $a$ is the acceleration in meters per
                      second squared ($m \cdot s^{-2}$)
            \end{itemize}
            This equation is fundamental in physics.
        \end{document}
    \end{minted}
\end{frame}

\begin{frame}[fragile]{Inline Equations}
    According to Newton's second law, the force applied to an object is given by $F = ma$, where:
    \begin{itemize}
        \item $F$ is the force in Newtons ($N$)
        \item $m$ is the mass in kilograms ($kg$)
        \item $a$ is the acceleration in meters per second squared ($m \cdot s^{-2}$)
    \end{itemize}

    This equation is fundamental in physics.
\end{frame}

\section{Tables}

\begin{frame}{Comparing Tables: \texttt{booktabs} vs. Regular Tables}
    \begin{center}
        Which table looks better?
    \end{center}
    \begin{columns}[T] % Align columns at the top
        \begin{column}{0.48\textwidth}
            \begin{table}
                \centering
                \caption{Regular style table.}
                \vspace{0.5em}
                \begin{tabular}{|c|c|c|}
                    \hline
                    \textbf{Parameter} & \textbf{Value} & \textbf{Unit} \\
                    \hline
                    Length & 10 & m \\
                    \hline
                    Width  &  5 & m \\
                    \hline
                    Height &  3 & m \\
                    \hline
                \end{tabular}
            \end{table}
        \end{column}
        \begin{column}{0.48\textwidth}
            \begin{table}
                \centering
                \caption{Booktabs style table.}
                \begin{tabular}{ccc}
                    \toprule
                    \textbf{Parameter} & \textbf{Value} & \textbf{Unit} \\
                    \midrule
                    Length & 10 & m \\
                    Width  &  5 & m \\
                    Height &  3 & m \\
                    \bottomrule
                \end{tabular}
            \end{table}
        \end{column}
    \end{columns}
\end{frame}

\begin{frame}{Why booktabs?}
    \begin{itemize}
        \item Developed to improve the quality of tables in \LaTeX \ documents
        \item Aimed at enhancing readability and visual appeal of tables.
        \item \textbf{Best Practices:}
            \begin{itemize}
                \item Avoid using vertical lines; rely on spacing and horizontal rules
                \item Maintain consistency in table design throughout the document
            \end{itemize}
    \end{itemize}
\end{frame}

\begin{frame}[fragile]{Basic Tables}
    \begin{minted}[fontsize=\scriptsize]{tex}
        \documentclass[12pt]{article}
        \usepackage{cleveref}
        \usepackage{booktabs}

        \begin{document}
            \begin{table}
                \centering
                \caption{Basic table.}
                \label{tab:basic}
                \begin{tabular}{l|cr}
                    \toprule
                    \textbf{Parameter} & \textbf{Value} & \textbf{Unit} \\
                    \midrule
                    Length             & 10             & m             \\
                    Width              &  5             & m             \\
                    \bottomrule
                \end{tabular}
            \end{table}
            \Cref{tab:basic} is a basic table.
        \begin{end}
    \end{minted}
\end{frame}

\begin{frame}[fragile]{Basic Tables}
    \begin{itemize}
        \item \mintinline{tex}{\begin{table}} ... \mintinline{tex}{\end{table}}
        \begin{itemize}
            \item Environment that holds the caption, label, and the actual table
        \end{itemize}
        \item \mintinline{tex}{\begin{tabular}} ... \mintinline{tex}{\end{tabular}}
        \begin{itemize}
            \item Defines the table's structure
            \item Column alignment specified within curly braces:
                \begin{itemize}
                    \item \texttt{l}: Left-aligned
                    \item \texttt{c}: Center-aligned
                    \item \texttt{r}: Right-aligned
                    \item \texttt{|}: Vertical lines (use sparingly with \texttt{booktabs})
                \end{itemize}
        \end{itemize}
    \item \mintinline{tex}{&} separates columns
    \item \mintinline{tex}{\\} ends rows
    \end{itemize}
\end{frame}

\begin{frame}[fragile]{Basic Tables}
    \begin{itemize}
        \item \mintinline{tex}{\toprule}: Thick line at the top
        \item \mintinline{tex}{\midrule}: Thin line between header and data
        \item \mintinline{tex}{\bottomrule}: Thick line at the bottom
    \end{itemize}
\end{frame}

\begin{frame}{Basic Tables}
    \begin{table}
        \centering
        \caption{Basic table.}
        \label{tab:basic}
        \begin{tabular}{l|cr}
            \toprule
            \textbf{Parameter} & \textbf{Value} & \textbf{Unit} \\
            \midrule
            Length & 10 & m \\
            Width  &  5 & m \\
            \bottomrule
        \end{tabular}
    \end{table}
    \Cref{tab:basic} is a basic table.
\end{frame}

\begin{frame}{Basic Tables -- Challenge}
    \begin{table}[ht]
        \begin{tabular}{l|cc}
        \toprule
            \textbf{Law} & \textbf{Differential form} & \textbf{Integral form}   \\ \midrule
            Gauss            & $\nabla \cdot \overset{\rightarrow}{\textbf{E}} = \frac{\rho}{\epsilon_0}$            & $\int_S \overset{\rightarrow}{\textbf{E}}\cdot\hat{n}\;\mathsf{dS}=\frac{\textbf{Q}}{\epsilon_0}$        \\
            Gauss for Magnetism            & $\nabla \cdot \overset{\rightarrow}{\textbf{B}} = 0$          & $\int_S \overset{\rightarrow}{\textbf{B}}\cdot\hat{n}\;\mathsf{dS} = 0$ \\
            Faraday            & $\nabla \times \overset{\rightarrow}{\textbf{E}}=- \frac{\partial \overset{\rightarrow}{\textbf{B}}}{\partial t}$            & $\oint \overset{\rightarrow}{\textbf{E}}\cdot\mathsf{d \overset{\rightarrow}{r}} = - \frac{d}{dt} \int_S \overset{\rightarrow}{\textbf{B}}\cdot\hat{n}\;\mathsf{dS}$ \\
            Ampère            & $\nabla \times \overset{\rightarrow}{\textbf{B}}=\mu_0\overset{\rightarrow}{\textbf{J}}+\mu_0\epsilon_0\frac{\partial \overset{\rightarrow}{\textbf{E}}}{\partial t}$            & $\oint \overset{\rightarrow}{\textbf{B}}\cdot\mathsf{d \overset{\rightarrow}{r}} = \mu_0\textbf{I} + \mu_0 \epsilon_0\int_S \overset{\rightarrow}{\textbf{E}}\cdot\hat{n}\;\mathsf{dS}$ \\ \bottomrule
        \end{tabular}
    \end{table}
\end{frame}

\begin{frame}[fragile]{Advanced Tables}
    \begin{itemize}
        \item As you probably can conclude, tables are not \LaTeX \ strong suit
        \item \url{https://www.tablesgenerator.com}
    \end{itemize}
\end{frame}

\section{Text Structure}

% Frame introducing Chapters, Sections, Subsections, and Subsubsections
\begin{frame}[fragile]{Text Structure}
    \begin{itemize}
        \item \textbf{Chapter} (\mintinline{tex}{\chapter{Chapter Name}}) - \textit{(Only in certain document classes)}
        \item \textbf{Section} (\mintinline{tex}{\section{Section Name}})
        \item \textbf{Subsection} (\mintinline{tex}{\subsection{Subsection Name}})
        \item \textbf{Subsubsection} (\mintinline{tex}{\subsubsection{Subsubsection Name}})
        \item It is possible to associate a label to each of these, for referencing
    \end{itemize}
\end{frame}

\begin{frame}[fragile]{Text Structure}
    \label{sl:structure}
    \begin{minted}[fontsize=\scriptsize]{tex}
        \documentclass{article}
        \usepackage{cleveref}

        \begin{document}
            \section{Introduction}
            \label{sec:intro}
            This is the introduction section.

            \section{Methodology}
            \label{sec:method}
            Refer to \cref{sec:intro} for the introduction.

            \subsection{Data Collection}
            \label{sec:data}
            Details about data collection for \cref{sec:method}.

            \subsubsection{Survey Design}
            \label{sec:survey}
            Information on survey design.
        \end{document}
    \end{minted}
\end{frame}

\section{Templates and Citations}

\begin{frame}[fragile]{IEEE Conference Template}
    \begin{itemize}
        \item \url{https://www.overleaf.com/latex/templates/ieee-conference-template/grfzhhncsfqn}
        \item Open as Template
    \end{itemize}
\end{frame}

\begin{frame}[fragile]{Organising a Project}
    \begin{itemize}
        \item Erase all pdfs
        \item Change \mintinline{tex}{conference_101719.tex} to \mintinline{tex}{main.tex}
    \item Erase everything between \mintinline{tex}{\end{IEEEkeywords}} and \mintinline{tex}{\end{document}}
    \end{itemize}
\end{frame}

\begin{frame}[fragile]{Organising a Project}
    \begin{itemize}
        \item Create a new file called \mintinline{tex}{body.tex}
        \item Paste the code in slide \ref{sl:structure} inside the \mintinline{tex}{document} environment into it
        \item Write \mintinline{tex}{\input{body.tex}} bellow \mintinline{tex}{\end{IEEEkeywords}}
        \begin{itemize}
            \item ´´Pastes'' the content of a tex file inside another
            \item We can easily write the content in \mintinline{tex}{body.tex} and change template, by using \mintinline{tex}{\input{body}}
        \end{itemize}
    \end{itemize}
\end{frame}

\begin{frame}[fragile]{Citations and Bibtex}
    \begin{itemize}
        \item Create a new file called \mintinline{tex}{citations.bib}
        \item Add the following two lines, below the \mintinline{tex}{\input{body}} line
    \end{itemize}
    \begin{minted}{tex}
        % This is the citations style
        \bibliographystyle{IEEEtran}
        % This is the file where the citations are stored
        \bibliography{citations}
    \end{minted}
\end{frame}

\begin{frame}[fragile]{Citations and Bibtex}
    \begin{itemize}
        \item Paste bibtex references inside \mintinline{tex}{citations.bib}
    \end{itemize}
    \begin{minted}[fontsize=\scriptsize]{bib}
        @INPROCEEDINGS{Filipe2021,
          author={Filipe, Jose N. and Carreira, J. and Tavora, Luis M. N.
                  and de Faria, Sergio M. M. and Navarro, Antonio and Assuncao, Pedro A. A.},
          booktitle={2021 Telecoms Conference (ConfTELE)},
          title={Tree-Based Ensemble Methods for Complexity Reduction of VVC Intra Coding},
          year={2021},
          volume={},
          number={},
          pages={1-6},
          doi={10.1109/ConfTELE50222.2021.9435476}
        }
    \end{minted}
    \begin{itemize}
        \item \mintinline{tex}{Filipe2021} is the key we are going to use to cite this workshop
        \item It can be changed to whatever we want
    \end{itemize}
\end{frame}

\begin{frame}[fragile]{Citations and Bibtex}
    \begin{itemize}
        \item Use \mintinline{tex}{\cite{...}} to cite a work
        \item For instance, add the following to \mintinline{tex}{body.tex}
    \end{itemize}
    \begin{minted}{tex}
        This work~\cite{Filipe2021} is not about \LaTeX.
    \end{minted}
\end{frame}

\begin{frame}[fragile]{Citations and Bibtex}
    \begin{itemize}
        \item Order in \mintinline{tex}{bib} file does not matter
        \item References are order according with the rules in the bibliography style (order of appearance, alphabetically, etc)
        \item Numbers are dynamically assigned by \LaTeX \ no need to worry about them at all!!!
    \end{itemize}
\end{frame}

\begin{frame}[fragile]{IPL Template}
    \begin{itemize}
        \item \url{https://www.overleaf.com/latex/templates/polytechnic-university-of-leiria-thesis-template/tqgbrncfhwgt}
        \item Open as Template
    \end{itemize}
\end{frame}

\section{Presentations and Beamer}

\begin{frame}{What is Beamer?}
    \begin{itemize}
        \item \textbf{Beamer} is a \LaTeX \ class for creating presentations
        \item It allows you to create slides with consistent styling and structure
        \item Utilizes frames to organize content into individual slides
    \end{itemize}
\end{frame}

\begin{frame}[fragile]{Basic Presentation}
    \begin{minted}{tex}
        \documentclass{beamer}

        \begin{document}

            \begin{frame}{Slide Title}
                Your content goes here.
            %\end{frame}

        \end{document}
    \end{minted}
\end{frame}

\begin{frame}[fragile]{Slightly Less Basic Presentation}
    \begin{minted}[fontsize=\scriptsize]{tex}
        \documentclass{beamer}
        \usetheme{Madrid}

        \title{Introduction to LaTeX Beamer}
        \author{Jane Doe}
        \institute{Engineering Department, XYZ University}
        \date{\today}

        \begin{document}

            \begin{frame}
                \titlepage
            % \end{frame}

            \begin{frame}{Slide Title}
                Your content goes here.
            % \end{frame}

        \end{document}
    \end{minted}
\end{frame}

\section{Conclusion}

{
\usebackgroundtemplate{\includegraphics[width=\paperwidth]{template/beamer_sps_final.pdf}}

\begin{frame}
\centering
{\Huge\usebeamercolor[fg]{frametitle}\textbf{Thank You}}
\end{frame}
}


\end{document}
